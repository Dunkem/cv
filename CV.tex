% !TEX TS-program = xelatex
% !TEX encoding = UTF-8 Unicode
% -*- coding: UTF-8; -*-
% vim: set fenc=utf-8

%%%%%%%%%%%%%%%%%%%%%%%%%%%%%%%%%%%%%%%%%%%%%%%%%%%%%%%%%%%%%%%%%
%% CV.tex
%% <https://github.com/zachscrivena/simple-resume-cv>
%% This is free and unencumbered software released into the
%% public domain; see <http://unlicense.org> for details.
%%%%%%%%%%%%%%%%%%%%%%%%%%%%%%%%%%%%%%%%%%%%%%%%%%%%%%%%%%%%%%%%%

% See "README.md" for instructions on compiling this document.

\documentclass[letterpaper,MMMyyyy,nonstopmode]{simpleresumecv}
% Class options:
% a4paper, letterpaper, nonstopmode, draftmode
% MMMyyyy, ddMMMyyyy, MMMMyyyy, ddMMMMyyyy, yyyyMMdd, yyyyMM, yyyy

%%%%%%%%%%%%%%%%%%%%%%%%%%%%%%%%%%%%%%%%%%%%%%%%%%%%%%%%%%%%%%%%%
%% PREAMBLE.
%%%%%%%%%%%%%%%%%%%%%%%%%%%%%%%%%%%%%%%%%%%%%%%%%%%%%%%%%%%%%%%%%

% CV Info (to be customized).
\newcommand{\CVAuthor}{Xiangyong Wen}
\newcommand{\CVTitle}{Xiangyong's CV}
\newcommand{\CVNote}{CV compiled on {\today}}

% PDF settings and properties.
\hypersetup{
pdftitle={\CVTitle},
pdfauthor={\CVAuthor},
pdfcreator={XeLaTeX},
pdfproducer={},
pdfkeywords={},
unicode=true,
bookmarks=true,
bookmarksopen=true,
pdfstartview=FitH,
pdfpagelayout=OneColumn,
pdfpagemode=UseOutlines,
hidelinks,
breaklinks}

% Shorthand.
\newcommand{\Code}[1]{\mbox{\textbf{#1}}}
\newcommand{\CodeCommand}[1]{\mbox{\textbf{\textbackslash{#1}}}}

%%%%%%%%%%%%%%%%%%%%%%%%%%%%%%%%%%%%%%%%%%%%%%%%%%%%%%%%%%%%%%%%%
%% ACTUAL DOCUMENT.
%%%%%%%%%%%%%%%%%%%%%%%%%%%%%%%%%%%%%%%%%%%%%%%%%%%%%%%%%%%%%%%%%

\begin{document}

%%%%%%%%%%%%%%%
% TITLE BLOCK %
%%%%%%%%%%%%%%%

\Title{\CVAuthor}

\begin{SubTitle}
Yibin, Sichuan, China
\par
\href{mailto:1127395857@qq.com}
{1127395857@qq.com}
\,\SubBulletSymbol\,
+86-18576757213
\,\SubBulletSymbol\,
\end{SubTitle}

\begin{Body}

%%%%%%%%%%%%%%%
%% EDUCATION %%
%%%%%%%%%%%%%%%

\Section
{Education}
{Education}
{PDF:Education}

\Entry
{\textbf{Southwest Petroleum University}},
Chengdu, Sichuan, China

\Gap
\BulletItem
B.S. in
\href{http://www.swpu.edu.cn/jdy/}
{School of Mechanical Engineering}
\hfill
\DatestampYMD{2013}{09}{06} --
\DatestampYMD{2018}{06}{21}
\begin{Detail}
\SubBulletItem
Final year project:Path planning and Tracking of Autonomous Navigation Robot Based on Dynamic Target Detection
\SubBulletItem
Graduated with Outstanding Thesis Award.
\SubBulletItem
GPA(Grade Point Average): 2.81/5.0
\end{Detail}

%%%%%%%%%%%%%%%%%%%%%%%%%%%
%% Research Interests %%
%%%%%%%%%%%%%%%%%%%%%%%%%%%

\Section
{Research Interests}
{Research Interests}
{PDF:Research Interests}

\Entry
\begin{Detail}
Personally, I am very keen on working with robot. Due to the limited resources of the school, 
I took a year out and then entered the DJI as an intern. During the internship, 
from machine vision to robot path planning, the direction of research and development was gradually clarified, 
also responsible for the path planning and decision-making of the RoboRTS framework. 
In the future, I hope that the planning of the robot involves more semantics, making the planning more intelligent and more in line with the target task.
\end{Detail}

%%%%%%%%%%%%%%%%%%%%%%%%%
%% Work EXPERIENCE %%
%%%%%%%%%%%%%%%%%%%%%%%%%

\Section
{Work Experience}
{Work Experience}
{PDF:Work Experience}

\Entry
\href{https://www.robomaster.com}
{\textbf{DJI Algorithm Engineer}}
Shenzhen, Guangdong, China
\hfill
\DatestampYMD{2018}{08}{30} --
Now
\begin{Detail}
    Improve the RoboRTS framework. Participate in company projects, in this project I extract 3d-tof information, generate 2d cost maps to verify navigation party in a complex environment. 
    And now plan to use the livox radar fusion vins to generate dense point clouds (hope to be familiar with 3d perception before starting 3d planning).
\end{Detail}

%%%%%%%%%%%%%%%%%%%%%%%%%%%
%% Internship Experience %%
%%%%%%%%%%%%%%%%%%%%%%%%%%%

\Section
{Internship Experience}
{Internship Experience}
{PDF:Internship Experience}

\Entry
\href{https://www.robomaster.com}
{\textbf{DJI Algorithm Engineer}}
Shenzhen, Guangdong, China
\hfill
\DatestampYMD{2016}{07}{03} --
\DatestampYMD{2018}{08}{29}

\hfill
\Entry
\href{https://github.com/RoboMaster/RoboRTS}
{\textbf{RoboRTS}}
\hfill
\DatestampYMD{2017}{08}{31} --
\DatestampYMD{2018}{05}{01}
\begin{Detail}
    RoboRTS is an open source software stack for Real-Time Strategy research on mobile robots, 
    developed by RoboMaster. The motivation for this project is RoboMaster AI Challenge. 
    In this robot challenge, multiple robots should fight with each other on a game field automatically. 
    It would be very convenient to have a unified framework for them to integrate hardware components and implement algorithms. 
    This project is a team work with four algorithm engineers. 
    My work in this project are things about motion planning and decision.
\end{Detail}

\BigGap

\Entry
\textbf{Projectile Trajectory Fitting}
\hfill
\DatestampYMD{2016}{12}{01} --
\DatestampYMD{2017}{05}{31}
\begin{Detail}
    In order to improve the shooting accuracy of the RoboMaster mobile robot, I establish a projectile
    trajectory fitting system, which can track projectile with binocular and then
    obtain the trajectory of the projectile. After analyzing the trajectory, verify the Projectile equation, it can  then control the gimbal of the robot
\end{Detail}

\BigGap

\Entry
\textbf{Armor Detection}
\hfill
\DatestampYMD{2016}{09}{10} --
\DatestampYMD{2016}{10}{31}
\begin{Detail}
    Detecting 2016 RoboMaster's robot armor, then calculate the position of the armor relative
     to the camera and send to the STM32 to control the gimbal. 
\end{Detail}

%%%%%%%%%%%%%%%%%%%%%%%
%% CAMPUS ACTIVITIES %%
%%%%%%%%%%%%%%%%%%%%%%%

\Section
{Campus Activities}
{Campus Activities}
{PDF:CampusActivities}

\Entry
\textbf{Robotics Club},
Southwest Petroleum University

\Gap
\BulletItem
Project Manager ( 1 year) \& Group Leader ( 1 year) 
\hfill
\DatestampYMD{2014}{12}{15} --
\DatestampYMD{2016}{06}{15}
\begin{Detail}
\SubBulletItem
Organizing the Club, push the progress of project.
\SubBulletItem
Machine design for mobile robot with cero. Also write the control logical of the robot.
\end{Detail}

\Entry
\textbf{Award}

\Gap
\BulletItem
National First Prize in RoboMaster 2015


%%%%%%%%%%%%%%%
%% LANGUAGES %%
%%%%%%%%%%%%%%%

\Section
{Languages}
{Languages}
{PDF:Languages}

\BulletItem
Chinese: Native language.

\Gap
\BulletItem
English: CET-4.

%%%%%%%%%%%%%%%%%%%%%%%
%% Programming Skills %%
%%%%%%%%%%%%%%%%%%%%%%%

\Section
{Work Skills}
{Work Skills}
{PDF:Work Skills}

\BulletItem
C++,
C, 
Python,
Matlab,

%%%%%%%%%%%%
%% SKILLS %%
%%%%%%%%%%%%

\Section
{Other Skills}
{Other Skills}
{PDF:Other Skills}

\Entry
cero, 
CAD,

\end{Body}

%%%%%%%%%%%
% CV NOTE %
%%%%%%%%%%%

% \BigGap
% \UseNoteFont%
\null\hfill%
% [\textit{\CVNote}]

\end{document}
